\documentclass{article}
\usepackage{graphicx} % Required for inserting images
\usepackage{geometry}

\title{Variational dating of ancestral recombination graphs}
\author{Nathaniel S. Pope, Yan Wong, Andrew Kern, Peter Ralph, others?}
\date{}

\begin{document}

\maketitle

\section{Introduction} (todo: Peter)
\begin{itemize}
\item What are ancestral recombination graphs, what information do they contain, how are they inferred.
\item Why do we care about dating vertices (or improve existing dating of vertices) in a scalable way? Tree sequences as concise, biologically informative data format. 
\item Why do we care about capturing/propagating uncertainty? How can we do so in a concise way? (eg for downstream analysis)
\end{itemize}

Notes:
argweaver: does it but can't store it compactly even if it could scale
compare to relate (which sorta has uncertainty by having diff dates for the "same" node)

\section{Model and variational approximation}
\begin{itemize}
\item Mutational model (Poisson; \textbf{figure:} describe/viz area below a node)
\item Gamma mixture model for prior (global prior; two or three gammas; updated a few times; explain first with just one (?))
\item Factorization into prior and edge likelihoods
\item Fully-factorized gamma variational approximation
\end{itemize}

\section{Expectation propagation}
\label{methods}
\begin{itemize}
\item Need some criterion for ``goodness of fit" of variational approximation $\rightarrow$ KL divergence; intractability of global minimization problem
\item Divide-and-conquer; background on expectation propagation
\item Specific form of edge updates
\item EM algorithm for updating mixture prior
\item Least squares correction for branch length constraint to get point estimates
\item KL minimization is exact for single trees -- that suggests two ways of using the algorithm: go tree by tree but map mutations across trees (Relate); work on the whole ARG (tsinfer)
\item Variational distribution for mutation ages
\end{itemize}


\section{Simulation experiment}
\begin{itemize}
\item Performance on simulated ARG (mutation ages, matched haplotypes, coverage, coalescent rates)
\item Performance on inferred ARGs (vs old tsdate, Relate, ARGNeedle).
\item Runtime / complexity
\end{itemize}


\section{Downstream analyses}
Split up between section \ref{methods} and section \ref{results}
\begin{itemize}
\item Coalescent rate estimation via EM : example of propagating uncertainty (in local estimation vs sequence-wide estimation)
\item Molecular dating of ancient samples : example from simulation? real data?
\item Marginal likelihood of mutation data conditioned on ARG topology (e.g. node age integrated out of model) -- this gives a residual
\item Stratify by mutation type / mutation rate / genomic coordinate
\item Date some famous mutations? Show coverage/accuracy across increasing numbers of samples, making point that scalability is useful [using unified genealogies]
\end{itemize}

\section{Discussion}
\begin{itemize}
\item Reiterate scalability / uncertainty bit
\item General framework for node-based inference tasks - emphasize needs only moments, could get these any way
\item Extensions include traits/geography, graph segmentation.
\end{itemize}

\section{Supplement}
\begin{itemize}
\item More background on EP (e.g. what are we minimizing and why does it work? KL minimization via moment matching, replace global KL divergence via Bethe entropy)
\item Derivation of EP updates (e.g. moment matching)
\item Implementation of ${_2}F_1$ with gradient
\item Derivation of EM updates for mixture prior
\item Derivation of mutation age distribution
\item Node matching algorithm
\item Downsampling polytomies
\end{itemize}
\end{document}
